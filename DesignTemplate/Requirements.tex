% !TEX root = DesignDoc.tex

\chapter{Requirements}
\label{chap:requirements}

As demonstration robots, the SMP and Mecanum robots don't have any specific external requirements. However, there were a few goals in mind when the robots were reassembled this more recent time:

\begin{itemize}
  \item{Stable Research Platform}
  \item{Good Examples of ROS Design}
  \item{Separate Noisy Components}
  \item{Easily Reconfigurable}
  \item{Easily Maintainable and Accessible}
\end{itemize}

\section{Stable Research Platform}

The typical sensor payload for the SMP and Mecanum robots includes:

\begin{itemize}
  \item{ASUS RGBD Vision Sensor}
  \item{LiDAR}
  \item{IMU}
\end{itemize}

The mechanical and electrical setup for the SMP robots currently includes the LiDAR, IMU, and ASUS Sensors, although the current ROS packages compiled on the robots do not use them.

\section{Good Examples of ROS Design}

Since the Mecanum and SMP robots are intended to be good examples of ROS design, great care was taken to use as many ROS Good Practices as possible. This includes using the system services model for auto-boot, udev rules for mounting hardware, dedicated driver nodes, launch files, and not using any custom messages. This is detailed further in Chapter \ref{chap:rosarchitecture}.

\section{Separate Noisy Components}

One of the problems these robots experienced in the past was a condition in which the motor controllers would unmount from the odroid. The cause was ruled to be noise generated by either the motor controllers themselves or the motors creating voltage spikes that scared the Odroid hardware into unmounting the USB devices to protect itself. To mitigate this problem, two measures were taken. First, motor controllers were selected which could be connected over serial lines. The big advantage here is that putting an FTDI between the motor controller and the Odroid adds a layer of electrical isolation. In addition, according to the documentation on the motor controller, the serial lines are more electrically stable than the USB lines. The USB was intended by the manufacturer for debugging only. The second measure taken was to keep all of the unregulated battery power inside the metal frame of the robot. All regulated 5V and 12V power is on the top.

\section{Easily Reconfigurable}

For all three robtots, one of the 

\section{Easily Maintainable and Accessible}

One of the big goals of the design was to make it easy to swap batteries out for charging, and also to make it as painless as posssible to make modifications. As it turned out, the robot frame for the SMPs ended up being pretty cramped, but 
