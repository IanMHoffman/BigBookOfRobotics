% !TEX root = HowToRobot.tex

\chapter{The Robot}
\label{chap:therobot}

\section{What Is A Robot?}

We've talked a lot so far about kinematics, not lighting yourself on fire, and getting ROS up and running, but what about all the bits that make your robot a robot? This is where things get tricky. There isn't one solution for all robots. I'm going to restrict this guide to RC wheeled robots and talk about some of the basic components you're going to need.

\section{Locomotion}

\subsection{Brushed DC Motors}

The typical drive motor. DC motors spin when power is applied, and spin the other way when power is applied with reverse polarity. DC motors aren't fancy, but they give you a good mix of speed and power. Almost every robot I have worked with used DC motors for motion. So, how do you choose a motor? That's actually pretty tricky, but there are a few numbers that are important.

\subsubsection{Torque}

Put very, very simply, torque is a measure of the strength of the motor. The units are weird, because torque is actually a measure of the force a motor can extert at a given distance from its center of rotation. More specifically, it is the cross product of force times distance, so you end up with things like newton-meters or ounce-inches.

But how much torque do you actually need? That involves some pretty hard math, actually. Your best bet is to look at the motors used on the SMP and Mecanum robots and judge based on those numbers. You can find those details in the design document that should be provided with this document.

\subsubsection{Voltage}

Remember when we talked about voltage? It's back! Brushed DC motors are odd in that they can actually handle being given higher or lower voltages than they are rated for. In fact, most good motors come with voltage vs torque curves for multiple supply voltages. Typically the datasheet will give absolute maximums and minimums for safe operation. The rated voltage is the one at which you will get the optimal tradeoff between torque and lifespan. Increasing the voltage will generally increase the torque and decreased the lifespan. Decreasing the voltage does the opposite. Of course, increasing the voltage too much will overheat the motor and destroy it. Remember Ohm's law? Increasing the voltage increases the current through the motor coil, and more current creates more heat.

\subsubsection{Current}

This one is a bit tricky conceptually, but easy to deal with in practice. Motors have an internal resistance, often called coil resistance. Ohm's law states that, given some resistance and some voltage, you will get some fixed current. However, in practice, the current draw actually depends on how hard the motor is being worked. The maximum current draw from a motor occurs when it is being driven constantly without being able to turn. This is called the stall current, and is generally listed in the motor documentation. For the sake of safety, assume the motors will draw a little more than their stall current.

\subsubsection{RPM}

Revolutions Per Minute. This is basically the speed at which the output shaft of (generally speaking) the gearbox will turn when the motor is spinning with no resistance. When you load it down with a motor, you can expect this number to drop considerably unless the motor has a phenomenal amount of torque.

\subsubsection{Gear Ratio}

Most motors come with a gearbox. In fact, often you will find a whole bunch of motors for sale that differ only in the gearbox. This makes sense, because making a different gearbox is way easier than making a different motor. The gearbox will have some gear-ratio, which probably means a bunch of really interesting things to people more mechanically inclined than I am. The easy version is that the gear-ratio is the number of times the motor shaft turns before one full revolution of the output shaft. A higher gear-ratio gives more torque, but reduces the actual speed of the output shaft.

\subsection{Brushless DC Motors}

Brushless DC motors are actually quite different in operation compared to Brushed DC motors, and require different control hardware. They are typically used in quad-copters and other applications that require extremely high RPM. They are also used in some performance RC cars. Typically, brushless motors do not have gearboxes because the wear at those speeds would make them explode, but there are exceptions.

I don't actually know what happens when you give a brushless motor more or less voltage than it expects. Never tried it. I would do some research though. Brushless motors can be temperamental.

\subsection{Stepper Motors}

Stepper motors work fundementally differently from DC motors. I won't get in to the details - the internet will do better by far - but to boil it down, stepper motors step instead of spinning continuously. You will most certainly want a dedicated driver board for a stepper motor, as the power has to be supplied to the magnetic coils in a very precise manner to get the most out of the stepper motor. Steppers may or may not have gearboxes, but adding a gearbox in this case both increases torque and the resolution of the steps while decreasing the maximum turning speed.

To drive a stepper motor, you generally hold a direction pin - DIR - high or low for clockwise or counter-clockwise (whether clockwise is high or low may vary based on the driver hardware) and then pulse a STEP pin to advance the motor by one step. This is very, very useful in situations that require a high degree of precision. You don't need additional hardware to tell exactly how far the motor has turned - in theory. In my experience, stepper motors have the potential to be forced to spin when commanded to stay still. There is no way to know when this has happened without feedback. However, if you can prevent that from happening by properly sizing the motor to the job, they are very accurate. If you are going to do something that requires a high degree of precision, like a 3D printer, stepper motors are an excellent solution.

\subsection{Servos}

Servos deserve a mention here because they are a very simple budget option for small robots. Continuous rotation servos can deliver a modest amount of power and speed, and take very little work to power or control. Servos are generally slower and weaker than DC or stepper motors, and are not terribly accurate. Feedback of some sort is essential, whether its in the form of encoders, line-following, or a vision system.

\section{Power}

Powering a robot is tricky business. We talked about batteries in Section \ref{sec:batteries}, which should get you at least as that far, but there are more considerations to be made.

\subsection{Regulators}
\label{sec:regulators}

Regulators are your friend. They keep your parts from exploding. A regulator takes a potentially noisy input (i.e. one that has spikes and voltage fluctuations) and smooths it. If there's a spike it can't handle, the regulator dies valiantly protecting the rest of your robot. This is a good thing. Regulators are much cheaper than, for example, a LiDAR.

\subsection{DC-DC Converters}

DC-DC converters take one voltage and convert it to another voltage. This is pretty cool. It means your 24V batteries can power your 3.3V cpu. It is worth noting that, because of the way transformers work, you can actually get more current out the low-voltage side than you put in. Of course, the total power usage is the same. Why exactly this happens is a topic for Circuits 2, but Equation \ref{eqn:converterpower} shows a handy formula to demonstrate how the converter deals with power in and out, where E is the efficiency of the converter. This is a value you can find in the datasheet. The voltages are, in general, fixed, as is the efficiency. So you can predict the high-voltage current draw based on what you expect the low-voltage current draw to be.

\begin{equation}
V_{in}*I_{in}*E=V_{out}*I_{out}
\label{eqn:converterpower}
\end{equation}

\subsection{}