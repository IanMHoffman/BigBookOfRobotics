% !TEX root = Proposal.tex

\chapter{Project Results}
\label{chap:results}

\section{Initial Conditions}

One thing that we noticed (and expected) is that the long-term behavior of the simulation is very sensitive to the game configuration parameters. For example, in our early simulations we discovered that the metabolic cost of eyes compared to the benefit of having the eyes would often lead to eyes being abandoned altogether in the first handful of generations.

\section{Eyes}

Of the two major aspects of this project, evolving behaviors was perhaps the more interesting. Unfortunately, we didn't see interesting behaviors emerging as regularly or as quickly as we hoped. We attribute this to two main factors.

First, eyes are expensive, but also required for complex behavior. This is somewhat realistic, but leads to being trapped in a local maxima with no eyes. Evolving good behaviors seems to take longer than just evolving the eyes away, so most simulations end up reducing to the minimum allowable number of eyes. We theorize that if we were to allow mutation and crossover on the Neural Nets with greater frequency than the characteristics, the Critters might be forced to evolve behaviors rather than just reducing metabolic cost by dropping eyes.

Second, the Critters are stupid. The ANN we used for behavior is very simple and has no concept of memory or context. We theorize that with a more complex Neural Net, more complex behaviors could emerge.

Although the eyes weren't always used effectively, some behavior that did somewhat commonly emerge was very nimbly avoiding other Critters (and sometimes food).

\section{Populations}

One of our original goals was to demonstrate stratification among different populations in the same environment. In other words, we wanted to create a situation in which there wasn't one "correct" genome. Rather, we wanted an environment in which the "best" configuration for each population depended on the current configuration of the other populations. We didn't see as much of this as we hoped.

We didn't see this as much as we had hoped, although it did happen on occasion. This is in large part because the Critters commonly all reduce to low metabolic cost states, which tend to be small Critters with the minimum number of eyes.

\section{Motion}
A popular method for many critters, particularly those with no eyes, was to spin in tight circles. We theorize this is because it would minimize chances of combat. Critters with eyes were also able to more completely observe their immediate surroundings. Occasionally a critter would evolve to have a lot speed and to travel a wide circle. This made them likely to ram into others, killing them and gaining life. These Critters also had more success at randomly eating food because they covered more ground. 

\section{Aggression}
We tried really hard to set up a configuration that would lead to a predatory population. However, this requires characteristics and behaviors to evolve together. In addition, since Critters don't know anything about the Critters they're observing, they have no idea if the target is weaker or stronger than they are. It was almost always safer to just avoid confrontation, even if it meant they didn't get the rather large life and health bonus for scoring a kill. Adding some more information about the enemy sighted to the neural net would likely allow the Critters to evolve more complex predatory behaviors.

