% !TEX root = Proposal.tex

\chapter{Evolution}
\label{chap:reproduction}

Like many evolutionary systems, the critters evolve through reproduction and mutation so their species lives on. To determine who is resurrected and who is replaced for the next round, the critters are ranked. Their rank is determined by how long they lived. Theoretically, the longer they live, the better suited to the environment they are and so, their ranking is higher. Anti-vaccers have disproven this theory but we're going to roll with it. After the ranking, the top 50\% of the population is kept for the next round and used as Parents for the crossover mutation, discussed later. The bottom 50\% is replaced with the 'children' of the top 50\%. \\
\newline
It should be noted that all non-eye characteristics such as attack, defense, etc. are handled in one reproductive function and all eye characteristics such as number, field of view, etc. are handled in a different function. This is because all non-eye characteristics are handled as doubles and the number of them do not change, while the eye characteristics involve dynamic memory due to the variation of the number of eyes. \\
\newline
To simulate a real critter reproducing, the crossover and mutation genetic algorithms were used to create the next generation of critters. Each were applied in two different ways to handle the non-eye characteristics and eye characteristics. 

\section{Crossover}
The crossover method, takes two parents, picks a random spot to cut their 'DNA' and splices the first chunk from Parent A to the second chunk from Parent B. This method works well for the non-eye characteristics, which are held in an indexable array. A random number in the array index range is chosen as the 'cut point'. Then the characteristics before that index are copied from Parent A and the characteristics after and including the index are copied from Parent B. \\

For the eyes, handling crossover is much trickier. To imitate real life, the child would have a chance at inheriting either Parent A's eye, Parent B's eye or no eye at all for the minimum number of eyes present. Then, the child would have a chance of inheriting or not inheriting the eye from the parent with more eyes. For example, if Parent A had two eyes and Parent B had three eyes, for the first two eyes, the characteristics would come from Parent A or Parent B. For the third eye, the child would either inherit the eye from Parent B or no eye at all. 


\section{Mutation}
The mutation method simply perturbs the current values by small amounts. In the configuration file, there are two user definable parameters for mutation: Mutation Rate and Mutation Chance. The Mutation Rate is how much a parameter can mutate. The Mutation Chance is the chance a characteristic will mutate. For the non-eye characteristics, a random number is generated for each characteristic and if that beats the Mutation chance, it is mutated. It's mutated with equation ~\ref{mutate-eqn}, where percent is a value between -1 and 1. 

\begin{equation}
val_{new} = val_{old} + (percent) (mutation_{rate})
\end{equation} \label{mutate-eqn}


The eyes are once again, a special case. They are mutated in two loops. First, the number of eyes is given a chance to mutate, which is user defined in the configuration file. The critter can either loose an eye or gain an eye. Then, the characteristics associated with the eye are mutated using the same methodology as the non-eye characteristics. 





